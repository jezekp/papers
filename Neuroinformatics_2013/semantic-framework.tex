%%%%%%%%%%%%%%%%%%%%%%% file template.tex %%%%%%%%%%%%%%%%%%%%%%%%%
%
% This is a general template file for the LaTeX package SVJour3
% for Springer journals.          Springer Heidelberg 2010/09/16
%
% Copy it to a new file with a new name and use it as the basis
% for your article. Delete % signs as needed.
%
% This template includes a few options for different layouts and
% content for various journals. Please consult a previous issue of
% your journal as needed.
%
%%%%%%%%%%%%%%%%%%%%%%%%%%%%%%%%%%%%%%%%%%%%%%%%%%%%%%%%%%%%%%%%%%%
%
% First comes an example EPS file -- just ignore it and
% proceed on the \documentclass line
% your LaTeX will extract the file if required
\begin{filecontents*}{example.eps}
%!PS-Adobe-3.0 EPSF-3.0
%%BoundingBox: 19 19 221 221
%%CreationDate: Mon Sep 29 1997
%%Creator: programmed by hand (JK)
%%EndComments
gsave
newpath
  20 20 moveto
  20 220 lineto
  220 220 lineto
  220 20 lineto
closepath
2 setlinewidth
gsave
  .4 setgray fill
grestore
stroke
grestore
\end{filecontents*}
%
\RequirePackage{fix-cm}
%
%\documentclass{svjour3}                     % onecolumn (standard format)
%\documentclass[smallcondensed]{svjour3}     % onecolumn (ditto)
%\documentclass[smallextended]{svjour3}       % onecolumn (second format)
\documentclass[twocolumn]{svjour3}          % twocolumn
%
\smartqed  % flush right qed marks, e.g. at end of proof
%
\usepackage{graphicx}

\usepackage[american]{babel}

%It ensures a direct compilation from tex to pdf.
\usepackage{epstopdf}
\usepackage[sort, numbers, authoryear]{natbib}
%
% \usepackage{mathptmx}      % use Times fonts if available on your TeX system
%
% insert here the call for the packages your document requires
%\usepackage{latexsym}
% etc.
%
% please place your own definitions here and don't use \def but
% \newcommand{}{}
%
% Insert the name of "your journal" with
 \journalname{Neuroinformatics}
%
\begin{document}

\title{Semantic Framework\thanks{This work was supported by the European Regional Development Fund (ERDF), Project "NTIS - New Technologies for Information Society", European Centre of Excellence, CZ.1.05/1.1.00/02.0090.}
}
\subtitle{Mapping Object Oriented Structures to Semantic Web Technologies}

%\titlerunning{Short form of title}        % if too long for running head

\author{Petr Je\v{z}ek         \and
        Second Author %etc.
}

%\authorrunning{Short form of author list} % if too long for running head

\institute{P. Je\v{z}ek \at
              New Technologies for the Information Society \\
              Department of Computer Science and Engineering\\
              Faculty of Applied Sciences \\
              University of  West Bohemia \\
              Univerzitn� 8 \\
              306 14  Plze\v{n} \\
              Czech Republic \\
              \email{jezekp@ntis.zcu.cz}           %  \\
%             \emph{Present address:} of F. Author  %  if needed
           \and
           S. Author \at
              second address
}

\date{Received: date / Accepted: date}
% The correct dates will be entered by the editor


\maketitle

\begin{abstract}
Insert your abstract here. Include keywords, PACS and mathematical
subject classification numbers as needed.
\keywords{First keyword \and Second keyword \and More}
% \PACS{PACS code1 \and PACS code2 \and more}
% \subclass{MSC code1 \and MSC code2 \and more}
\end{abstract}

\section{Introduction}
\label{intro}
Our research group specializes in the research of brain activity. The main interest of experiments is a measurement of reaction time e. g. drivers. We use methods of Electroencephalography (EEG) and its subdomain Event-Related Potentials (ERP). Measurement of EEG/ERP experiments is a time consuming task that produces a lot of data. With the increasing number of experiments their long-term storage has to be solved. Although many laboratories deal with data collection their long term storage is not satisfactorily solved. The most pressing difficulties is an absence of standardized generally used data formats. The data formats should describe not only raw data but researchers still more emphasize a need to have a well-defined metadata description of experimental data. As a suitable way to describe a specific domains seems to be its description by domain ontology.

There were formed several organizations that aim is to solve difficulties with sustainability of neuroscience databases. Probably the most important of them is International Neuroinformatics Coordinating Facility (INCF) formed in 2005 \citep{INCF-evaluation-first-year}. INCF is being developing an infrastructure serves to national nodes developing a partial infrastructure of each specific domain.

The suitable medium for sharing experimental data should be the Internet. Despite the popularity of the Internet, how it is growing, it contains of a huge amount of information with  practically no classification. Such not classified data are not suitable for sharing. As a solution an extension called the Semantic Web, that is intended to give the data semantic meaning, is being developed. Ontology is considered one of the pillars of the Semantic Web.

As a result, suitable ontology that precisely describes domain of EEG/ERP experiments should help with its interpretation. However current software systems are usually object-oriented and semantic gaps between object-oriented code and Semantic Web languages are significant. We present a proposed mapping and its implementation that extends common object-oriented code by missing semantics. The proposed mapping we practically implemented within a custom Semantic Framework.

We developed EEG/ERP Portal \citep{ISI:000306821100004} intended to manage and share experimental data\footnote{http://eegdatabase.kiv.zcu.cz}. The internal structure of EEG/ERP portal is designed with respecting INCF recommendations \citep{incf-sustainability-report}. Integration of developed Framework within EEG/ERP Portal enables transformation of stored experiments into the Semantic Web structure. This structure provides access to data thought Neuroscience Information Framework (NIF) \citep{NIF-Neuroinformatics}.

\section{Systems for Data Modeling}
\label{Systems for Data Modeling}

\subsection{Common Systems}
\label{Common_Systems}

Several ways to represent data in electrophysiology exist. Since raw data has to be extended by description metadata existing modeling techniques has to be investigated.  After analyzing several semantic data models \citep{0720407583} for years still only  essentially two data modeling formalism stay: \emph{entity-relationship-attribute (ERA)} and \emph{object-oriented (OO)} models.

While relational databases are based on relational model represented by a collection of data items organized as a set of formally-described tables object-oriented modeling describes the abstract syntax of modeling concepts, their attributes and relationships.

\subsection{Semantic Web Modeling}
\label{Semantic_Web_Modeling}

When World Wide Web (WWW) is the largest knowledge database available for human readers it consists of a huge amount of information with practically no classification. It is extremely difficult to handle this enormous amount of information. Although search engines as Google, Yahoo or Alta Vista several important disadvantages stays. Mainly, they have low precision, low or no recall, results are highly sensitive to vocabulary and results are single web pages.

One possible solution to deal with these difficulties is to develop increasingly sophisticates techniques based on artificial intelligence and computational linguistics. An alternative, probably most promising, approach is to represent web content in a form that is more easily machine-processable and to use intelligent techniques to take advantage of these representation. One of these approaches is the Semantic Web \citep{Semantic_Web_Primer}.


\section{Mapping Data Models}
\label{Mapping_Data_Models}

\subsection{Data concepts}
\label{Data_Concepts}
The main construct in object-oriented modeling (OOM) is an object. Since entities and their relations exist in relational databases the similar representation exist in OOM as well. Entities are represented by objects and its relations are represented by their association ends represented by class attributes. All objects should be unique so every object has an identity. This identity is represented by an object identifier (OID) that distinguishes it from all other objects. Similar mechanism exist in relational databases as well. 

Object-oriented concepts are described by UML diagrams. Object oriented programming languages use the concepts in practical sense. 

\subsection{Concepts Comparison}
\label{concepts_comparison}
\emph{Ontology Definition Metamodel}\citep{OMG2009} compares concepts of \emph{Web Ontology Language (OWL)} with the features of UML. It compares the features the two have in common with the features in one but not the other. 

\subsubsection{Similar Concepts}
\label{Similar_concepts}

Both UML and OWL are based on classes. While extent of UML classes is on or more instances, in OWL the extent of class is a set of individuals defined by URIs. Because all individuals are subclasses of an universal class \emph{Thing} and individuals may be instances of \emph{Thing} and not necessary to any other class, it can be used outside the system.

Relationship among OWL classes are called properties. Properties are two types. \emph{ObjectProperty} if the type of property is another individual or \emph{DataTypeProperty} when the property is an atomic type. Both UML and OWL support separation into modules, called \emph{package} in UML and \emph{Ontology} in OWL. Both support enumeration of elements.

OWL properties can be constrained by cardinality restrictions. Cardinality restrictions in UML properties are limited to \emph{many-to-many}, \emph{many-to-one} or \emph{one-to-one} relationships. 
 
OWL allows properties to be declared symmetric or transitive. UML uses \emph{Object Constraint Language (OCL)} \citep{DBLP:conf/sfm/CabotG12}. UML classifiers are \emph{private}, \emph{protected}, \emph{default} or \emph{public} while OWL classifiers are always \emph{public}.

\subsubsection{Different Concepts}
\label{Different_concepts}


\section{Semantic Framework}
\label{Semantic_Framework}

\subsection{Proposed Mapping}
\label{Proposed_Mapping}

\section{Related Work}
\label{Related_Work}

\section{Discussion}
\label{discussion}

\section{Future Work}
\label{Future_Work}

\section{Conclusion}
\label{Conclusion}

\section{Information Sharing Statement}
\label{Information_sharing_statement}


\subsection{Subsection title}
\label{sec:2}
as required. Don't forget to give each section
and subsection a unique label (see Sect.~\ref{sec:1}).
\paragraph{Paragraph headings} Use paragraph headings as needed.
\begin{equation}
a^2+b^2=c^2
\end{equation}

% For one-column wide figures use
\begin{figure}
% Use the relevant command to insert your figure file.
% For example, with the graphicx package use
  \includegraphics{example.eps}
% figure caption is below the figure
\caption{Please write your figure caption here}
\label{fig:1}       % Give a unique label
\end{figure}
%
% For two-column wide figures use
\begin{figure*}
% Use the relevant command to insert your figure file.
% For example, with the graphicx package use
  \includegraphics[width=0.75\textwidth]{example.eps}
% figure caption is below the figure
\caption{Please write your figure caption here}
\label{fig:2}       % Give a unique label
\end{figure*}
%
% For tables use
\begin{table}
% table caption is above the table
\caption{Please write your table caption here}
\label{tab:1}       % Give a unique label
% For LaTeX tables use
\begin{tabular}{lll}
\hline\noalign{\smallskip}
first & second & third  \\
\noalign{\smallskip}\hline\noalign{\smallskip}
number & number & number \\
number & number & number \\
\noalign{\smallskip}\hline
\end{tabular}
\end{table}


%\begin{acknowledgements}
%If you'd like to thank anyone, place your comments here
%and remove the percent signs.
%\end{acknowledgements}

% BibTeX users please use one of
%\bibliographystyle{apalike}  % used another style because provided styles don't work
\bibliographystyle{spbasic}      % basic style, author-year citations
%\bibliographystyle{spmpsci}      % mathematics and physical sciences
%\bibliographystyle{spphys}       % APS-like style for physics
\bibliography{semantic-framework}   % name your BibTeX data base


\end{document}
% end of file template.tex

